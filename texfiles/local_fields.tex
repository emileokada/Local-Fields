\documentclass{memoir}

% Imports

%% Quotations (S. Gammelgaard)
\usepackage{verbatim}
\usepackage{csquotes}

%% Mathematics
\usepackage{amsfonts}
\usepackage{amsmath}
\usepackage{amssymb}    % Extra symbols
\usepackage{amsthm}     % Theorem-like environments
\usepackage{calligra}	% For the \sheafHom command
\usepackage{cancel}     % Cancel terms with \cancel, \bcancel or \xcancel
\usepackage{dsfont}     % Double stroke font with \mathds{}
\usepackage{mathtools}  % Fonts and environments for mathematical formulae
\usepackage{mathrsfs}   % Script font with \mathscr{}
\usepackage{stmaryrd}   % Brackets
\usepackage{thmtools}   % Theorem-like environments, extends amsthm

%% Graphics
\usepackage[dvipsnames,svgnames,cmyk]{xcolor}     % Pre-defined colors
\usepackage{graphicx}         % Tool for importing images
\graphicspath{{figures/}}
\usepackage{tikz}             % Drawing tool
\usetikzlibrary{calc}
\usetikzlibrary{intersections}
\usetikzlibrary{decorations.markings}
\usetikzlibrary{arrows}
\usetikzlibrary{positioning}
\usepackage{tikz-cd}		  % Commutative diagrams
\usepackage[all]{xy}

%% Organising tools
\usepackage[notref, notcite]{showkeys}               % Labels in margins
\usepackage[color= LightGray,bordercolor = LightGray,textsize    = footnotesize,figwidth    = 0.99\linewidth,obeyFinal]{todonotes} % Marginal notes

%% Misc
\usepackage{xspace}         % Clever space
\usepackage{textcomp}       % Extra symbols
\usepackage{multirow}       % Rows spanning multiple lines in tables
\usepackage{tablefootnote}  % Footnotes for tables

%% Bibliography
\usepackage[backend = biber, style = alphabetic, ibidtracker=true]{biblatex}
\addbibresource{bibliography.bib}

%% Cross references
\usepackage{varioref}
%\usepackage[nameinlink, capitalize, noabbrev]{cleveref}
\usepackage[pdftex,hidelinks]{hyperref}

\pageaiv
\stockaiv

\setsecnumdepth{subsection}

\pretitle{\begin{center}\huge\sffamily\bfseries}

%% Book
\renewcommand*{\printbooktitle}[1]
{
    \hrule\vskip\onelineskip
    \centering\booktitlefont #1
    \vskip\onelineskip\hrule
}
\renewcommand*{\afterbookskip}{\par}
\renewcommand*{\booktitlefont}{\Huge\bfseries\sffamily}
\renewcommand*{\booknamefont}{\normalfont\huge\bfseries\MakeUppercase}


%% Part
\renewcommand*{\printparttitle}[1]
{
    \hrule\vskip\onelineskip
    \centering\parttitlefont #1
    \vskip\onelineskip\hrule
}
\renewcommand*{\afterpartskip}{\par}
\renewcommand*{\parttitlefont}{\Huge\bfseries\sffamily}
\renewcommand*{\partnamefont}{\normalfont\huge\bfseries\MakeUppercase}


%% Chapter 
\makeatletter
\chapterstyle{demo2}
\renewcommand*{\printchaptername}
{
    \centering\chapnamefont\MakeUppercase{\@chapapp}
}
\renewcommand*{\printchapternum}{\chapnumfont\thechapter\space}
\renewcommand*{\chaptitlefont}{\Huge\bfseries\sffamily\center}
\let\ps@chapter\ps@empty


%% Lower level sections
\setsecheadstyle{\Large\bfseries\sffamily\raggedright}
\setsubsecheadstyle{\large\bfseries\sffamily\raggedright}
\setsubsubsecheadstyle{\normalsize\bfseries\sffamily\raggedright}
\setparaheadstyle{\normalsize\bfseries\sffamily\raggedright}
\setsubparaheadstyle{\normalsize\bfseries\sffamily\raggedright}


%% Abstract
\renewcommand{\abstractnamefont}{\sffamily\bfseries}


%% Header
\pagestyle{ruled}
\makeevenhead{ruled}{\sffamily\leftmark}{}{}
\makeoddhead {ruled}{}{}{\sffamily\rightmark}


%% Trim marks
\trimLmarks

%% Environments
\declaretheorem[style = plain, numberwithin = section]{thm}
\declaretheorem[style = plain,      sibling = thm]{corollary}
\declaretheorem[style = plain,      sibling = thm]{lemma}
\declaretheorem[style = plain,      sibling = thm]{proposition}
\declaretheorem[style = plain,      sibling = thm]{observation}
\declaretheorem[style = plain,      sibling = thm]{conjecture}
\declaretheorem[style = definition, sibling = thm]{definition}
\declaretheorem[style = definition, sibling = thm]{example}
\declaretheorem[style = definition, sibling = thm]{notation}
\declaretheorem[style = remark,     sibling = thm]{remark}
%\crefname{observation}{Observation}{Observations}
%\Crefname{observation}{Observation}{Observations}
%\crefname{conjecture}{Conjecture}{Conjectures}
%\Crefname{conjecture}{Conjecture}{Conjectures}
%\crefname{notation}{Notation}{Notations}
%\Crefname{notation}{Notation}{Notations}
%\crefname{diagram}{Diagram}{Diagrams}
%\Crefname{diagram}{Diagram}{Diagrams}

%% Operators
\DeclareMathOperator{\spn}{span}				% Span of vectors
\DeclareMathOperator{\Gal}{Gal}					% Galois group
\DeclareMathOperator{\Spec}{Spec}				% Spectrum
\DeclareMathOperator{\Proj}{Proj}				% Proj construction
\DeclareMathOperator{\Gr}{\mathbb{G}}			% Grassmannian
\DeclareMathOperator{\Aut}{Aut}					% Automorphisms
\DeclareMathOperator{\End}{End}					% Endomorphisms
\DeclareMathOperator{\CH}{CH}					% Chow ring/group
\DeclareMathOperator{\CHr}{CH^\bullet}			% Chow ring
\DeclareMathOperator{\Cox}{Cox}					% Cox ring
\DeclareMathOperator{\Div}{Div}					% Divisor group
\DeclareMathOperator{\Cl}{Cl}					% Class group
\DeclareMathOperator{\Pic}{Pic}					% Picard group
\DeclareMathOperator{\relSpec}{\mathbf{Spec}}	% Relative Spec
\DeclareMathOperator{\relProj}{\mathbf{Proj}}	% Relative Proj
\DeclareMathOperator{\ord}{ord}					% Order
\DeclareMathOperator{\res}{res}					% Residue
\DeclareMathOperator{\coker}{coker}				% Cokernel (\ker is already defined)
\DeclareMathOperator{\im}{im}					% Image
\DeclareMathOperator{\coim}{coim}			    % Coimage
\DeclareMathOperator{\tr}{tr}					% Trace
\DeclareMathOperator{\rk}{rk}					% Rank
\DeclareMathOperator{\Hom}{Hom}					% Homomorphisms
\DeclareMathOperator{\cl}{cl}					% Class map
\DeclareMathOperator{\sheafHom}					% Sheaf of homomorphisms
{
    \mathscr{H}\text{\kern -5.2pt {\calligra\large om}}\,
}
\DeclareMathOperator{\codim}{codim}				% Codimension
\DeclareMathOperator{\Sym}{Sym}					% Symmetric powers
\DeclareMathOperator{\II}{I\!I}					% Second fundamental form
\DeclareMathOperator{\Pfaff}{Pfaff}				% Pfaffian

%% Delimiters
\DeclarePairedDelimiter{\p}{\lparen}{\rparen}          % Parenthesis
\DeclarePairedDelimiter{\set}{\lbrace}{\rbrace}        % Set
\DeclarePairedDelimiter{\abs}{\lvert}{\rvert}          % Absolute value
\DeclarePairedDelimiter{\norm}{\lVert}{\rVert}         % Norm
\DeclarePairedDelimiter{\ip}{\langle}{\rangle}         % Inner product, ideal
\DeclarePairedDelimiter{\sqb}{\lbrack}{\rbrack}        % Square brackets
\DeclarePairedDelimiter{\ssqb}{\llbracket}{\rrbracket} % Double brackets
\DeclarePairedDelimiter{\ceil}{\lceil}{\rceil}         % Ceiling
\DeclarePairedDelimiter{\floor}{\lfloor}{\rfloor}      % Floor
\DeclarePairedDelimiter{\tuple}{\langle}{\rangle}		% Tuple	


%% Sets
\newcommand{\N}{\mathbb{N}}    						% Natural numbers
\newcommand{\Z}{\mathbb{Z}}    						% Integers
\newcommand{\Q}{\mathbb{Q}}    						% Rational numbers
\newcommand{\R}{\mathbb{R}}    						% Real numbers
\newcommand{\C}{\mathbb{C}}    						% Complex numbers
\newcommand{\A}{\mathbb{A}}    						% Affine space
\renewcommand{\P}{\mathbb{P}}  						% Projective space
%Additions (S. Gammelgaard)
\renewcommand{\H}{\mathbb{H}}						% Hyperbolic space, or half-plane
\newcommand{\D}{\mathbb{D}} 						% Unit disk
\newcommand{\F}{\mathbb{F}} 						% Field
\newcommand{\bP}[1]{\mathbf{P}\!\left(#1\right)}	% Projectivisation of bundles

%% Special groups and Lie groups
\newcommand{\GL}{\mathbf{GL}}						% General linear group
\newcommand{\PGL}{\mathbf{PGL}}						% Projective linear group
\newcommand{\SL}{\mathbf{SL}}						% Special linear group

%% Lie algebras
\newcommand{\lalg}[1]{{\normalfont\mathfrak{#1}}}	% General for Lie algebras
\newcommand{\gl}{\lalg{gl}}							% General linear algebra
%\newcommand{\sl}{\lalg{sl}}							% Special linear algebra

%% Cones of cycles on varieties and related objects
\newcommand{\NS}{\mathrm{NS}}						% Neron-Severi group
\newcommand{\Nef}{\mathrm{Nef}}						% Nef cone
\newcommand{\NE}{\mathrm{NE}}						% Cone of curves
\newcommand{\Eff}{\mathrm{Eff}}						% Effective cone
\newcommand{\Pseff}{\mathrm{PSeff}}					% Pseudoeffective cone

%% Categories
\newcommand{\cat}[1]{{\normalfont\mathsf{#1}}}	% General for categories
\newcommand{\Cat}{\cat{Cat}}					% Category of categories
\newcommand{\Sch}{\cat{Sch}}					% Schemes
\newcommand{\Aff}{\cat{Aff}}					% Affine schemes
\newcommand{\Set}{\cat{Set}}					% Sets
\newcommand{\Grp}{\cat{Grp}}					% Groups
\newcommand{\AbGrp}{\cat{AbGrp}}				% AbGroups
\newcommand{\Ab}{\cat{Ab}}      				% AbGroups
\newcommand{\Ring}{\cat{Ring}}					% Rings
\newcommand{\Mod}{\cat{Mod}}				    % R-Modules
\newcommand{\Top}{\cat{Top}}					% Topological spaces
\newcommand{\SMan}{\cat{Man}^\infty}			% Smooth manifolds
\newcommand{\Coh}[1]{\cat{Coh}({#1})}			% Coherent sheaves
\newcommand{\QCoh}[1]{\cat{QCoh}({#1})}			% Quasi-coherent sheaves
\newcommand{\Fun}{\cat{Fun}}					% Category of functors
\newcommand{\PreSh}{\cat{PreSh}}			    % Category of presheaves
\newcommand{\Sh}{\cat{Sh}}			            % Category of presheaves

%% Miscellaneous mathematics
\newcommand{\ol}[1]{\overline{#1}}							% Overline
\newcommand{\Dirsum}{\bigoplus}								% Direct sum
\newcommand{\shf}[1]{\mathscr{#1}}							% Sheaf
\newcommand{\OO}{\mathcal{O}}								% Structure sheaf
\DeclareMathOperator{\id}{id}								% Identity
\newcommand{\tens}[1]{\otimes_{#1}}							% Tensor product
\newcommand{\normal}{\vartriangleleft}						% Normal subgroup, ideal of ring or Lie algebra
\newcommand{\lamron}{\vartriangleright}						% The opposite of above
\newcommand{\dvol}{d\operatorname{vol}}						% Volume form on a KÀhler manifold
\newcommand{\cha}{\operatorname{char}}						% Characteristic of a field
\newcommand{\Hilb}{\operatorname{Hilb}}						% Hilbert scheme
\newcommand{\isoto}{\xrightarrow{\sim}}						% Isomorphism
\newcommand{\injto}{\xhookrightarrow{}}						% Injective map
\newcommand{\ratto}{\dashrightarrow}						% Rational map
\newcommand{\rateq}{\overset{\sim}{\ratto}}					% Rational equivalence
\newcommand{\Bl}[2]{\operatorname{Bl}_{#2} #1}				% Blow-up of #1 along #2
%\newcommand{\Bl}[2]{#1\kern -2pt \uparrow_{#2}}			% 	(alternativ som ingen andre liker, buhu)
\newcommand{\fracpart}[2]{\frac{\partial #1}{\partial #2}}	% Partial derivative
\renewcommand{\setminus}{\smallsetminus}
\newcommand{\transp}[1]{{}^t#1}								% transposed map, Voisin-style
\newcommand{\dual}{{}^\vee}									% dual of map, vector bundle, sheaf, etc...
\newcommand{\littletilde}{\tilde}							% for the next
\renewcommand{\tilde}{\widetilde}
\newcommand{\Spe}{\text{Sp\'e}}						        % Etale space
\newcommand{\colim}{\text{colim}}						    % Colimit
\newcommand{\supp}{\text{supp}}						        % Support
\newcommand{\rad}{\text{rad}}						        % Radical of an ideal
\newcommand{\Ann}{\text{Ann}}						        % Annihilator
\newcommand{\Frac}{\text{Frac}}						        % Field of fractions

%%\newcommand{\dual}{{}^{\smash{\scalebox{.7}[1.4]{\rotatebox{90}{\guilsinglleft}}}}}	% Dual of sheaf/vector space et cetera

%% Miscellaneous, not-strictly-mathematical
\renewcommand{\qedsymbol}{\(\blacksquare\)}
\newcommand{\ie}{\leavevmode\unskip, i.e.,\xspace}
\newcommand{\eg}{\leavevmode\unskip, e.g.,\xspace}
%\newcommand{\wlog}{\leavevmode\unskip without loss of generality \xspace}
\newcommand{\dash}{\textthreequartersemdash\xspace}
\newcommand{\TikZ}{Ti\textit{k}Z\xspace}
\newcommand{\matlab}{\textsc{Matlab}\xspace}


\title{Local fields}
\author{Emile T. Okada}

\begin{document}
\maketitle
\tableofcontents
\chapter{Discrete valuation rings and Dedekind domains}
\section{Preliminaries}
\begin{thm}
    \label{thm:int_dep}
    Let $S\subseteq R$ be rings and $x\in R$.
    The following are equivalent:
    \begin{enumerate}
        \item $x$ is integral over $S$
        \item $S[x]$ is a finite $S$-algebra
        \item $S[x]$ is contained ina subring $T\subseteq R$ that is a finite $S$-alg
        \item There is an $S[x]$-module $M$ such that $\Ann_{S[x]}(M) = 0$ and $M$ is finitely generated as a $S$-module.
    \end{enumerate}
\end{thm}
\begin{lemma}
    Let $S\subseteq R$ be an integral extension of rings. 
    Let $Q\normal R$ be a prime ideal, and set $P = Q\cap S$.
    Then $Q$ is maximal in $R$ iff $P$ is maximal in $S$.
\end{lemma}
\begin{lemma}
    (Incomparability).
    Let $S\subseteq R$ be an integral extension of rings.
    Let $Q,Q'\normal R$ be prime ideals with $Q\subseteq Q'$ and $Q\cap S = Q'\cap S$. 
    Then $Q = Q'$.
\end{lemma}
\begin{lemma}
    (Lying over).
    Let $S\subseteq R$ be an integral extension of rings.
    Let $P\normal S$ be prime.
    Then there exists a prime $Q\normal R$ such that $Q\cap S = P$.
\end{lemma}
\begin{thm}
    \label{thm:going_up}
    (Going up).
    Let $S\subseteq R$ be an integral extension of rings.
    Let $P_1\subseteq P_2\subseteq \cdots \subseteq P_n$ be a chain of primes in $S$ and $Q_1\subseteq Q_2\subseteq \cdots \subseteq Q_m$ with $0\le m < n$ be a chain of primes in $R$ with $Q_i\cap S = P_i$.
    Then the chain can be extended to $Q_1\subseteq Q_2\subseteq \cdots \subseteq Q_n$ with $Q_i\cap S = P_i$.
\end{thm}
\begin{thm}
    \label{thm:going_down}
    (Going down).
    Let $S\subseteq R$ be an integral extension of domains such that $S$ is normal.
    Let $P_1\supseteq P_2\supseteq \cdots \supseteq P_n$ be a chain of primes in $S$ and $Q_1\supseteq Q_2\supseteq \cdots \supseteq Q_m$ with $0\le m < n$ be a chain of primes in $R$ with $Q_i\cap S = P_i$.
    Then the chain can be extended to $Q_1\supseteq Q_2\supseteq \cdots \supseteq Q_n$ with $Q_i\cap S = P_i$.
\end{thm}
\begin{thm}
    \label{thm:noeth_loc_ring}
    Let $R$ be a Noetherian local ring with maximal ideal $\mathfrak m$.
    Then $\cap_n\mathfrak m^n = 0$.
\end{thm}
\section{Discrete valuation rings}
\begin{definition}
    Let $K$ be a field.
    A \textit{discrete valuation} on $K$ is a surjective function $v:K\to \Z\cup\{\infty\}$ satisfying
    \begin{enumerate}
        \item $v(x) = \infty$ iff $x=0$
        \item $v(xy) = v(x) + v(y)$ 
        \item $v(x+y) \ge \inf(v(x),v(y))$.
    \end{enumerate}
    If $(K,v)$ is a field with a discrete valuation then $R = \{x\in K: v(x) \ge 0\}$ is a subring of $K$ called the valuation ring of $v$.
\end{definition}
\begin{definition}
    A domain $R$ is called a \textit{discrete valuation ring} if there is a valuation $v$ on $K=\text{Frac}(R)$ for which $R$ is the valuation ring.
\end{definition}
\begin{lemma}
    Any DVR is a Noetherian local domain of Krull dimension 1.
\end{lemma}
\begin{proof}
    Let $t\in R$ be such that $v(t) = 1$.
    Then all the ideals of $R$ are in the chain 
    \begin{equation}
        R \supsetneq (t) \supsetneq (t^2) \supsetneq (t^3) \supsetneq \cdots \supseteq 0.
    \end{equation}
\end{proof}
\begin{thm}
    Let $R$ be a Noetherian local domain with Krull dimenstion 1 and maximal ideal $\mathfrak m$.
    The following are equivalent
    \begin{enumerate}
        \item $R$ is a DVR
        \item $\mathfrak m$ is principal
        \item $R$ is normal.
    \end{enumerate}
\end{thm}
\begin{proof}
    From the conditions on $R$ we know the only prime ideals are $0$ and $\mathfrak m$.

    $(1)\Rightarrow (3)$ Let $x\in K = \text{Frac}(R)$ and suppose it is integral over $R$.
    Then there are $a_0,\dots,a_{n-1}\in R$ such that $x^n+a_{n-1}x^{n-1}+\cdots a_1x+a_0 = 0$.
    Suppose that $v(x)<0$. Then 
    \begin{equation}
        nv(x) = v(x^n) \ge \inf(v(a_ix^i)) \ge (n-1)v(x)
    \end{equation}
    and so $v(x)\ge0$ which is a contradiction and so $v(x)\ge0$ and hence $x\in r$.

    $(3)\Rightarrow (2)$ Let $0\ne x\in \mathfrak m$. Then $\rad((x)) = \mathfrak m$ and so since $R$ is Noetherian there is an $n\in \mathbb N$ such that $\mathfrak m^n\subseteq (x) \subseteq \mathfrak m$. Let $n$ be minimal with this property. 
    \begin{remark}
        Morally we now have $(x) = \mathfrak m^n$.
    \end{remark}
    Let $y \in \mathfrak m^{n-1}$ be such that $y\not\in (x)$.
    Set $z = x/y\in K$.
    Then $z^{-1}\not\in R$ so $z^{-1}$ is not integral over $R$.
    But this means that $z^{-1}\mathfrak m\nsubseteq \mathfrak m$ my theorem \ref{thm:int_dep}.
    But $y\mathfrak m\subset (x)$ and so $z^{-1}\mathfrak m\subseteq R$ and hence $z^{-1}\mathfrak m = R$  and so $\mathfrak m = (z)$.

    $(2)\Rightarrow (1)$ Define obvious valuation. 
\end{proof}
\begin{proposition}
    $R$ is a DVR iff it is a Noetherian local ring and its maximal ideal is generated by a non-nilpotent element $\pi$.
\end{proposition}
\begin{proof}
    By theorem \ref{thm:noeth_loc_ring} every element in $R$ can be written as $\pi^nu$ for $u$ a unit and $n\in \N_0$.
    It follows that $R$ must be an integral domain, and so this expression is unique.
    Setting $v(\pi^nu) = n$ one can easily check that this defines valuation on $\Frac(R)$ with valuation ring $R$.
\end{proof}
\section{Dedekind domains}
\begin{definition}
    A \textit{Dedekind domain} is a normal Noetherian domain of Krull dimension 1.
\end{definition}
\begin{lemma}
    Let $R$ be a Noetherian domain of Krull dimension 1.
    Then $R$ is a Dedekind domain iff every local ring $R_{\mathfrak p}$ for $\mathfrak p \ne 0$ is a DVR.
\end{lemma}
\begin{definition}
    Let $R$ be a Dedekind domain with $K=\Frac(R)$.
    We call an $R$-submodule of $K$ a fractional ideal if it is finitely generated as an $R$-module.
\end{definition}
\begin{remark}
    One can define the product of fractional ideals in the obvious way.
    Then the products of fractional ideals are fractional ideals.
\end{remark}
\begin{proposition}
    Let $R$ be a Dedekind domain. 
    Then all fractional ideals are invertible.
\end{proposition}
\subsection{Examples}
\subsubsection{PIDs}
\begin{thm}
    Let $R$ be a PID. Then $R$ is a Dedekind domain.
\end{thm}
\begin{proof}
    Certainly Noetherian.
    Also clearly has Krull dimension 1.
    Finally, $R_{\mathfrak p}$ is a local PID and hence a DVR for all $0\ne \mathfrak p\normal R$.
\end{proof}
\subsubsection{Rings of integers}
\begin{thm}
    Let $K/\Q$ be a number field and $\mathcal O_K$ its ring of algebraic integers.
    Then $\mathcal O_K$ is a Dedekind domain.
\end{thm}
\begin{proof}
    If $\alpha \in \Frac(\mathcal O_K) = K$ is integral over $\mathcal O_K$ then it is integral over $\Z$ and so lies in $\mathcal O_K$.
    Thus $\mathcal O_K$ is normal.
    To see that $\mathcal O_K$ is Noetherian note that it is finite over $\Z$ and hence Noetherian.
    Finally, $\mathcal O_K$ has Krull dimension 1 since it is integral over $\Z$ and $\Z$ has Krull dimension 1.
\end{proof}
\subsubsection{Coordinate rings}
\begin{thm}
    Let $V$ be an affine variety defined over an algebraically closed field $k$.
    Then $k[V]$ is a Dedekind domain iff $V$ is non-singular, irreducible and of dimension $1$.
\end{thm}
\begin{proof}
    $k[V]$ is always Noetherian.
    $V$ is irreducible iff $k[V]$ is a domain.
    $V$ is of dimension 1 iff $k[V]$ has Krull dimension 1.

    Now suppose $k[V]$ is a Dedekind domain. Then we know that all the local rings are normal Noetherian local domains  i.e. DVRs and hence $V$ must be non-singular.
    Conversely, suppose $V$ is non-singular. Then all the local rings of $k[V]$ are DVRs and so $k[V]$ is a Dedekind domain.
\end{proof}
\end{document}


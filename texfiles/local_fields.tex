\documentclass{memoir}

% Imports

%% Quotations (S. Gammelgaard)
\usepackage{verbatim}
\usepackage{csquotes}

%% Mathematics
\usepackage{amsfonts}
\usepackage{amsmath}
\usepackage{amssymb}    % Extra symbols
\usepackage{amsthm}     % Theorem-like environments
\usepackage{calligra}	% For the \sheafHom command
\usepackage{cancel}     % Cancel terms with \cancel, \bcancel or \xcancel
\usepackage{dsfont}     % Double stroke font with \mathds{}
\usepackage{mathtools}  % Fonts and environments for mathematical formulae
\usepackage{mathrsfs}   % Script font with \mathscr{}
\usepackage{stmaryrd}   % Brackets
\usepackage{thmtools}   % Theorem-like environments, extends amsthm

%% Graphics
\usepackage[dvipsnames,svgnames,cmyk]{xcolor}     % Pre-defined colors
\usepackage{graphicx}         % Tool for importing images
\graphicspath{{figures/}}
\usepackage{tikz}             % Drawing tool
\usetikzlibrary{calc}
\usetikzlibrary{intersections}
\usetikzlibrary{decorations.markings}
\usetikzlibrary{arrows}
\usetikzlibrary{positioning}
\usepackage{tikz-cd}		  % Commutative diagrams
\usepackage[all]{xy}

%% Organising tools
\usepackage[notref, notcite]{showkeys}               % Labels in margins
\usepackage[color= LightGray,bordercolor = LightGray,textsize    = footnotesize,figwidth    = 0.99\linewidth,obeyFinal]{todonotes} % Marginal notes

%% Misc
\usepackage{xspace}         % Clever space
\usepackage{textcomp}       % Extra symbols
\usepackage{multirow}       % Rows spanning multiple lines in tables
\usepackage{tablefootnote}  % Footnotes for tables

%% Bibliography
\usepackage[backend = biber, style = alphabetic, ibidtracker=true]{biblatex}
\addbibresource{bibliography.bib}

%% Cross references
\usepackage{varioref}
%\usepackage[nameinlink, capitalize, noabbrev]{cleveref}
\usepackage[pdftex,hidelinks]{hyperref}

\pageaiv
\stockaiv

\setsecnumdepth{subsection}

\pretitle{\begin{center}\huge\sffamily\bfseries}

%% Book
\renewcommand*{\printbooktitle}[1]
{
    \hrule\vskip\onelineskip
    \centering\booktitlefont #1
    \vskip\onelineskip\hrule
}
\renewcommand*{\afterbookskip}{\par}
\renewcommand*{\booktitlefont}{\Huge\bfseries\sffamily}
\renewcommand*{\booknamefont}{\normalfont\huge\bfseries\MakeUppercase}


%% Part
\renewcommand*{\printparttitle}[1]
{
    \hrule\vskip\onelineskip
    \centering\parttitlefont #1
    \vskip\onelineskip\hrule
}
\renewcommand*{\afterpartskip}{\par}
\renewcommand*{\parttitlefont}{\Huge\bfseries\sffamily}
\renewcommand*{\partnamefont}{\normalfont\huge\bfseries\MakeUppercase}


%% Chapter 
\makeatletter
\chapterstyle{demo2}
\renewcommand*{\printchaptername}
{
    \centering\chapnamefont\MakeUppercase{\@chapapp}
}
\renewcommand*{\printchapternum}{\chapnumfont\thechapter\space}
\renewcommand*{\chaptitlefont}{\Huge\bfseries\sffamily\center}
\let\ps@chapter\ps@empty


%% Lower level sections
\setsecheadstyle{\Large\bfseries\sffamily\raggedright}
\setsubsecheadstyle{\large\bfseries\sffamily\raggedright}
\setsubsubsecheadstyle{\normalsize\bfseries\sffamily\raggedright}
\setparaheadstyle{\normalsize\bfseries\sffamily\raggedright}
\setsubparaheadstyle{\normalsize\bfseries\sffamily\raggedright}


%% Abstract
\renewcommand{\abstractnamefont}{\sffamily\bfseries}


%% Header
\pagestyle{ruled}
\makeevenhead{ruled}{\sffamily\leftmark}{}{}
\makeoddhead {ruled}{}{}{\sffamily\rightmark}


%% Trim marks
\trimLmarks

%% Environments
\declaretheorem[style = plain, numberwithin = section]{thm}
\declaretheorem[style = plain,      sibling = thm]{corollary}
\declaretheorem[style = plain,      sibling = thm]{lemma}
\declaretheorem[style = plain,      sibling = thm]{proposition}
\declaretheorem[style = plain,      sibling = thm]{observation}
\declaretheorem[style = plain,      sibling = thm]{conjecture}
\declaretheorem[style = definition, sibling = thm]{definition}
\declaretheorem[style = definition, sibling = thm]{example}
\declaretheorem[style = definition, sibling = thm]{notation}
\declaretheorem[style = remark,     sibling = thm]{remark}
%\crefname{observation}{Observation}{Observations}
%\Crefname{observation}{Observation}{Observations}
%\crefname{conjecture}{Conjecture}{Conjectures}
%\Crefname{conjecture}{Conjecture}{Conjectures}
%\crefname{notation}{Notation}{Notations}
%\Crefname{notation}{Notation}{Notations}
%\crefname{diagram}{Diagram}{Diagrams}
%\Crefname{diagram}{Diagram}{Diagrams}

%% Operators
\DeclareMathOperator{\spn}{span}				% Span of vectors
\DeclareMathOperator{\Gal}{Gal}					% Galois group
\DeclareMathOperator{\Spec}{Spec}				% Spectrum
\DeclareMathOperator{\Proj}{Proj}				% Proj construction
\DeclareMathOperator{\Gr}{\mathbb{G}}			% Grassmannian
\DeclareMathOperator{\Aut}{Aut}					% Automorphisms
\DeclareMathOperator{\End}{End}					% Endomorphisms
\DeclareMathOperator{\CH}{CH}					% Chow ring/group
\DeclareMathOperator{\CHr}{CH^\bullet}			% Chow ring
\DeclareMathOperator{\Cox}{Cox}					% Cox ring
\DeclareMathOperator{\Div}{Div}					% Divisor group
\DeclareMathOperator{\Cl}{Cl}					% Class group
\DeclareMathOperator{\Pic}{Pic}					% Picard group
\DeclareMathOperator{\relSpec}{\mathbf{Spec}}	% Relative Spec
\DeclareMathOperator{\relProj}{\mathbf{Proj}}	% Relative Proj
\DeclareMathOperator{\ord}{ord}					% Order
\DeclareMathOperator{\res}{res}					% Residue
\DeclareMathOperator{\coker}{coker}				% Cokernel (\ker is already defined)
\DeclareMathOperator{\im}{im}					% Image
\DeclareMathOperator{\coim}{coim}			    % Coimage
\DeclareMathOperator{\tr}{tr}					% Trace
\DeclareMathOperator{\rk}{rk}					% Rank
\DeclareMathOperator{\Hom}{Hom}					% Homomorphisms
\DeclareMathOperator{\cl}{cl}					% Class map
\DeclareMathOperator{\sheafHom}					% Sheaf of homomorphisms
{
    \mathscr{H}\text{\kern -5.2pt {\calligra\large om}}\,
}
\DeclareMathOperator{\codim}{codim}				% Codimension
\DeclareMathOperator{\Sym}{Sym}					% Symmetric powers
\DeclareMathOperator{\II}{I\!I}					% Second fundamental form
\DeclareMathOperator{\Pfaff}{Pfaff}				% Pfaffian

%% Delimiters
\DeclarePairedDelimiter{\p}{\lparen}{\rparen}          % Parenthesis
\DeclarePairedDelimiter{\set}{\lbrace}{\rbrace}        % Set
\DeclarePairedDelimiter{\abs}{\lvert}{\rvert}          % Absolute value
\DeclarePairedDelimiter{\norm}{\lVert}{\rVert}         % Norm
\DeclarePairedDelimiter{\ip}{\langle}{\rangle}         % Inner product, ideal
\DeclarePairedDelimiter{\sqb}{\lbrack}{\rbrack}        % Square brackets
\DeclarePairedDelimiter{\ssqb}{\llbracket}{\rrbracket} % Double brackets
\DeclarePairedDelimiter{\ceil}{\lceil}{\rceil}         % Ceiling
\DeclarePairedDelimiter{\floor}{\lfloor}{\rfloor}      % Floor
\DeclarePairedDelimiter{\tuple}{\langle}{\rangle}		% Tuple	


%% Sets
\newcommand{\N}{\mathbb{N}}    						% Natural numbers
\newcommand{\Z}{\mathbb{Z}}    						% Integers
\newcommand{\Q}{\mathbb{Q}}    						% Rational numbers
\newcommand{\R}{\mathbb{R}}    						% Real numbers
\newcommand{\C}{\mathbb{C}}    						% Complex numbers
\newcommand{\A}{\mathbb{A}}    						% Affine space
\renewcommand{\P}{\mathbb{P}}  						% Projective space
%Additions (S. Gammelgaard)
\renewcommand{\H}{\mathbb{H}}						% Hyperbolic space, or half-plane
\newcommand{\D}{\mathbb{D}} 						% Unit disk
\newcommand{\F}{\mathbb{F}} 						% Field
\newcommand{\bP}[1]{\mathbf{P}\!\left(#1\right)}	% Projectivisation of bundles

%% Special groups and Lie groups
\newcommand{\GL}{\mathbf{GL}}						% General linear group
\newcommand{\PGL}{\mathbf{PGL}}						% Projective linear group
\newcommand{\SL}{\mathbf{SL}}						% Special linear group

%% Lie algebras
\newcommand{\lalg}[1]{{\normalfont\mathfrak{#1}}}	% General for Lie algebras
\newcommand{\gl}{\lalg{gl}}							% General linear algebra
%\newcommand{\sl}{\lalg{sl}}							% Special linear algebra

%% Cones of cycles on varieties and related objects
\newcommand{\NS}{\mathrm{NS}}						% Neron-Severi group
\newcommand{\Nef}{\mathrm{Nef}}						% Nef cone
\newcommand{\NE}{\mathrm{NE}}						% Cone of curves
\newcommand{\Eff}{\mathrm{Eff}}						% Effective cone
\newcommand{\Pseff}{\mathrm{PSeff}}					% Pseudoeffective cone

%% Categories
\newcommand{\cat}[1]{{\normalfont\mathsf{#1}}}	% General for categories
\newcommand{\Cat}{\cat{Cat}}					% Category of categories
\newcommand{\Sch}{\cat{Sch}}					% Schemes
\newcommand{\Aff}{\cat{Aff}}					% Affine schemes
\newcommand{\Set}{\cat{Set}}					% Sets
\newcommand{\Grp}{\cat{Grp}}					% Groups
\newcommand{\AbGrp}{\cat{AbGrp}}				% AbGroups
\newcommand{\Ab}{\cat{Ab}}      				% AbGroups
\newcommand{\Ring}{\cat{Ring}}					% Rings
\newcommand{\Mod}{\cat{Mod}}				    % R-Modules
\newcommand{\Top}{\cat{Top}}					% Topological spaces
\newcommand{\SMan}{\cat{Man}^\infty}			% Smooth manifolds
\newcommand{\Coh}[1]{\cat{Coh}({#1})}			% Coherent sheaves
\newcommand{\QCoh}[1]{\cat{QCoh}({#1})}			% Quasi-coherent sheaves
\newcommand{\Fun}{\cat{Fun}}					% Category of functors
\newcommand{\PreSh}{\cat{PreSh}}			    % Category of presheaves
\newcommand{\Sh}{\cat{Sh}}			            % Category of presheaves

%% Miscellaneous mathematics
\newcommand{\ol}[1]{\overline{#1}}							% Overline
\newcommand{\Dirsum}{\bigoplus}								% Direct sum
\newcommand{\shf}[1]{\mathscr{#1}}							% Sheaf
\newcommand{\OO}{\mathcal{O}}								% Structure sheaf
\DeclareMathOperator{\id}{id}								% Identity
\newcommand{\tens}[1]{\otimes_{#1}}							% Tensor product
\newcommand{\normal}{\vartriangleleft}						% Normal subgroup, ideal of ring or Lie algebra
\newcommand{\lamron}{\vartriangleright}						% The opposite of above
\newcommand{\dvol}{d\operatorname{vol}}						% Volume form on a KÀhler manifold
\newcommand{\cha}{\operatorname{char}}						% Characteristic of a field
\newcommand{\Hilb}{\operatorname{Hilb}}						% Hilbert scheme
\newcommand{\isoto}{\xrightarrow{\sim}}						% Isomorphism
\newcommand{\injto}{\xhookrightarrow{}}						% Injective map
\newcommand{\ratto}{\dashrightarrow}						% Rational map
\newcommand{\rateq}{\overset{\sim}{\ratto}}					% Rational equivalence
\newcommand{\Bl}[2]{\operatorname{Bl}_{#2} #1}				% Blow-up of #1 along #2
%\newcommand{\Bl}[2]{#1\kern -2pt \uparrow_{#2}}			% 	(alternativ som ingen andre liker, buhu)
\newcommand{\fracpart}[2]{\frac{\partial #1}{\partial #2}}	% Partial derivative
\renewcommand{\setminus}{\smallsetminus}
\newcommand{\transp}[1]{{}^t#1}								% transposed map, Voisin-style
\newcommand{\dual}{{}^\vee}									% dual of map, vector bundle, sheaf, etc...
\newcommand{\littletilde}{\tilde}							% for the next
\renewcommand{\tilde}{\widetilde}
\newcommand{\Spe}{\text{Sp\'e}}						        % Etale space
\newcommand{\colim}{\text{colim}}						    % Colimit
\newcommand{\supp}{\text{supp}}						        % Support
\newcommand{\rad}{\text{rad}}						        % Radical of an ideal
\newcommand{\Ann}{\text{Ann}}						        % Annihilator
\newcommand{\Frac}{\text{Frac}}						        % Field of fractions

%%\newcommand{\dual}{{}^{\smash{\scalebox{.7}[1.4]{\rotatebox{90}{\guilsinglleft}}}}}	% Dual of sheaf/vector space et cetera

%% Miscellaneous, not-strictly-mathematical
\renewcommand{\qedsymbol}{\(\blacksquare\)}
\newcommand{\ie}{\leavevmode\unskip, i.e.,\xspace}
\newcommand{\eg}{\leavevmode\unskip, e.g.,\xspace}
%\newcommand{\wlog}{\leavevmode\unskip without loss of generality \xspace}
\newcommand{\dash}{\textthreequartersemdash\xspace}
\newcommand{\TikZ}{Ti\textit{k}Z\xspace}
\newcommand{\matlab}{\textsc{Matlab}\xspace}


\title{Local fields}
\author{Emile T. Okada}

\begin{document}
\maketitle
\tableofcontents
\chapter{Discrete valuation rings and Dedekind domains}
\section{Discrete valuation rings}
\begin{definition}
    Let $K$ be a field.
    A \textit{discrete valuation} on $K$ is a surjective function $v:K\to \Z\cup\{\infty\}$ satisfying
    \begin{enumerate}
        \item $v(x) = \infty$ iff $x=0$
        \item $v(xy) = v(x) + v(y)$ 
        \item $v(x+y) \ge \inf(v(x),v(y))$.
    \end{enumerate}
    If $(K,v)$ is a field with a discrete valuation then $R = \{x\in K: v(x) \ge 0\}$ is a subring of $K$ called the valuation ring of $v$.
\end{definition}
\begin{definition}
    A domain $R$ is called a \textit{discrete valuation ring} if there is a valuation $v$ on $K=\text{Frac}(R)$ for which $R$ is the valuation ring.
\end{definition}
\begin{lemma}
    Any DVR is a Noetherian local domain of Krull dimension 1.
\end{lemma}
\begin{proof}
    Let $t\in R$ be such that $v(t) = 1$.
    Then all the ideals of $R$ are in the chain 
    \begin{equation}
        R \supsetneq (t) \supsetneq (t^2) \supsetneq (t^3) \supsetneq \cdots \supseteq 0.
    \end{equation}
\end{proof}
\begin{thm}
    Let $R$ be a Noetherian local domain with Krull dimenstion 1 and maximal ideal $\mathfrak m$.
    The following are equivalent
    \begin{enumerate}
        \item $R$ is a DVR
        \item $\mathfrak m$ is principal
        \item $R$ is normal.
    \end{enumerate}
\end{thm}
\begin{proof}
    From the conditions on $R$ we know the only prime ideals are $0$ and $\mathfrak m$.

    $(1)\Rightarrow (3)$ Let $x\in K = \text{Frac}(R)$ and suppose it is integral over $R$.
    Then there are $a_0,\dots,a_{n-1}\in R$ such that $x^n+a_{n-1}x^{n-1}+\cdots a_1x+a_0 = 0$.
    Suppose that $v(x)<0$. Then 
    \begin{equation}
        nv(x) = v(x^n) \ge \inf(v(a_ix^i)) \ge (n-1)v(x)
    \end{equation}
    and so $v(x)\ge0$ which is a contradiction and so $v(x)\ge0$ and hence $x\in r$.

    $(3)\Rightarrow (2)$ Let $0\ne x\in \mathfrak m$. Then $\rad((x)) = \mathfrak m$ and so since $R$ is Noetherian there is an $n\in \mathbb N$ such that $\mathfrak m^n\subseteq (x) \subseteq \mathfrak m$. Let $n$ be minimal with this property. 
    \begin{remark}
        Morally we now have $(x) = \mathfrak m^n$.
    \end{remark}
    Let $y \in \mathfrak m^{n-1}$ be such that $y\not\in (x)$.
    Set $z = x/y\in K$.
    Then $z^{-1}\not\in R$ so $z^{-1}$ is not integral over $R$.
    But this means that $z^{-1}\mathfrak m\nsubseteq \mathfrak m$ my theorem \ref{thm:int_dep}.
    But $y\mathfrak m\subset (x)$ and so $z^{-1}\mathfrak m\subseteq R$ and hence $z^{-1}\mathfrak m = R$  and so $\mathfrak m = (z)$.

    $(2)\Rightarrow (1)$ Define obvious valuation. 
\end{proof}
\begin{proposition}
    $R$ is a DVR iff it is a Noetherian local ring and its maximal ideal is generated by a non-nilpotent element $\pi$.
\end{proposition}
\begin{proof}
    By theorem \ref{thm:noeth_loc_ring} every element in $R$ can be written as $\pi^nu$ for $u$ a unit and $n\in \N_0$.
    It follows that $R$ must be an integral domain, and so this expression is unique.
    Setting $v(\pi^nu) = n$ one can easily check that this defines valuation on $\Frac(R)$ with valuation ring $R$.
\end{proof}
\subsubsection{Summary of useful properties of DVRs}
\begin{enumerate}
    \item If $a,b\in K = \Frac(R)$ have $\nu(a) \ne \nu (b)$ then $\nu(a+b) = \min(\nu(a),\nu(b))$.
\end{enumerate}
\section{Dedekind domains}
\begin{definition}
    A \textit{Dedekind domain} is a normal Noetherian domain of Krull dimension 1.
\end{definition}
\begin{lemma}
    Let $R$ be a Noetherian domain of Krull dimension 1.
    Then $R$ is a Dedekind domain iff every local ring $R_{\mathfrak p}$ for $\mathfrak p \ne 0$ is a DVR.
\end{lemma}
\subsection{Examples}
\subsubsection{PIDs}
\begin{thm}
    Let $R$ be a PID. Then $R$ is a Dedekind domain.
\end{thm}
\begin{proof}
    Certainly Noetherian.
    Also clearly has Krull dimension 1.
    Finally, $R_{\mathfrak p}$ is a local PID and hence a DVR for all $0\ne \mathfrak p\normal R$.
\end{proof}
\subsubsection{Rings of integers}
\begin{thm}
    Let $K/\Q$ be a number field and $\mathcal O_K$ its ring of algebraic integers.
    Then $\mathcal O_K$ is a Dedekind domain.
\end{thm}
\begin{proof}
    If $\alpha \in \Frac(\mathcal O_K) = K$ is integral over $\mathcal O_K$ then it is integral over $\Z$ and so lies in $\mathcal O_K$.
    Thus $\mathcal O_K$ is normal.
    To see that $\mathcal O_K$ is Noetherian note that it is finite over $\Z$ and hence Noetherian.
    Finally, $\mathcal O_K$ has Krull dimension 1 since it is integral over $\Z$ and $\Z$ has Krull dimension 1.
\end{proof}
\subsubsection{Coordinate rings}
\begin{thm}
    Let $V$ be an affine variety defined over an algebraically closed field $k$.
    Then $k[V]$ is a Dedekind domain iff $V$ is non-singular, irreducible and of dimension $1$.
\end{thm}
\begin{proof}
    $k[V]$ is always Noetherian.
    $V$ is irreducible iff $k[V]$ is a domain.
    $V$ is of dimension 1 iff $k[V]$ has Krull dimension 1.

    Now suppose $k[V]$ is a Dedekind domain. Then we know that all the local rings are normal Noetherian local domains  i.e. DVRs and hence $V$ must be non-singular.
    Conversely, suppose $V$ is non-singular. Then all the local rings of $k[V]$ are DVRs and so $k[V]$ is a Dedekind domain.
\end{proof}
\subsection{Properties of Dedekind domains}
\begin{enumerate}
    \item Consider $\Spec(R)$ where $R$ is a Dedekind domain.
    Then $\Spec(R)$ consists of closed points and a generic point.
    \item For any prime ideal $\mathfrak p\normal R$, we have natural inclusions $R\subseteq R_{\mathfrak p} \subseteq K = \Frac(R)$.
    \item If we interpret $K$ as the rational functions on $\Spec(R)$, then $\nu_{\mathfrak p}(k)$ is the degree of the pole/zero of $k$ at the point $\mathfrak p$.
        Rational functions on $\Spec(R)$ have finitely many zeros/poles.
    \item $\mathfrak p = R\cap (\mathfrak p R_{\mathfrak p})$.
    Thus for $r\in R$, $r\in \mathfrak p$ iff $\nu_{\mathfrak p}(r)\ge 1$.
    \item An element $k\in K$ is in $R$ iff $\nu_{\mathfrak p}(k)\ge 0$ for all $\mathfrak p\in\Spec(R)$.
\end{enumerate}
\subsubsection{Rigorous part}
\begin{definition}
    Let $R$ be a Dedekind domain with $K=\Frac(R)$.
    We call an $R$-submodule of $K$ a fractional ideal if it is finitely generated as an $R$-module.
\end{definition}
\begin{remark}
    One can define the product of fractional ideals in the obvious way.
    Then the products of fractional ideals are fractional ideals.
\end{remark}
\begin{lemma}
    \label{lemma:frac_DVR}
    Let $R$ be a DVR with uniformising element $\pi$. 
    Then all the non-zero fractional ideals of $R$ are of the form $\pi^nR$ for $n\in \Z$.
\end{lemma}
\begin{proof}
    Elements in $K = \Frac(R)$ can uniquely be written in the form $\pi^nu$ for $n\in\Z,u\in R^\times$.
    The fact that fractional ideals are finitely generated ensures that they have an element with minimal valuation.
    The result follows.
\end{proof}
\begin{thm}
    Let $R$ be a Dedekind domain. 
    Then all non-zero fractional ideals are invertible.
\end{thm}
\begin{proof}
    Let $0\ne \mathfrak a$ be a fractional ideal of $R$ and $\mathfrak a' = (R:\mathfrak a)$.
    It suffices to show that $\mathfrak a . \mathfrak a' = R$.
    But by lemma \ref{lemma:frac_DVR}, we have $\mathfrak a_{\mathfrak p} \mathfrak a'_{\mathfrak p} = R_{\mathfrak p}$ for all prime ideals $0\ne \mathfrak p \normal R$.
    If $\mathfrak p = 0$ then $\mathfrak a_{\mathfrak p} \mathfrak a'_{\mathfrak p} = R_{\mathfrak p}$ holds by inspection.
    But $\mathfrak a . \mathfrak a' \subseteq R$ always holds and so we must have equality.
\end{proof}
\begin{corollary}
    The set of non-zero fraction ideals of $R$ form a group under multiplication.
\end{corollary}
\begin{remark}
    Note that $(R:xR) = x^{-1}R$.
\end{remark}
\begin{lemma}
    \label{lemma:DCC}
    Let $x\in R\backslash 0$. Then the set of ideals containing $x$ satisfy the descending chain condition.
\end{lemma}
\begin{proof}
    Let $R\supseteq \mathfrak a_1\supseteq \mathfrak a_2 \supseteq \cdots \supseteq xR$.
    Then $xR \subseteq x\mathfrak a_1^{-1} \subseteq x\mathfrak a_2^{-1} \subseteq \cdots \subseteq R$.
    The result now follows from the fact that $R$ is a Noetherian domain.
\end{proof}
\begin{proposition}
    \label{prop:finite_prime}
    Let $x\in R\backslash 0$. Then only finitely many prime ideals contain $x$.
\end{proposition}
\begin{proof}
    Suppose $x\in \mathfrak p_1, \mathfrak p_2, \dots$.
    By lemma \ref{lemma:DCC}, $\mathfrak p_1\supseteq \mathfrak p_1\cap \mathfrak p_2 \supseteq \cdots$ must stabilise. 
    Thus there is an $n$ such that $\mathfrak p_n \supseteq \cap_{i=1}^{n-1}\mathfrak p_i$.
    But then there is an $i$ such that $1\le i < n$ and $\mathfrak p_n\supseteq \mathfrak p_i$.
    Since $R$ has Krull dimension 1 this forces $\mathfrak p_n = \mathfrak p_i$ and so we obtain a contradition.
\end{proof}
\begin{corollary}
    Let $\nu_{\mathfrak p}$ denote the valution on $K = \Frac(R)$ obtained from $R_{\mathfrak p}$ for $\mathfrak p \normal R$ a prime.
    Then for $0\ne x\in K$, the numbers $\nu_{\mathfrak p}(x)$ are almost all $0$.
\end{corollary}
\begin{definition}
    Let $\mathfrak a \normal R$ be a fractional ideal and let $\mathfrak p \normal R$ be a prime ideal. 
    Then $\mathfrak a_{\mathfrak p} = (\mathfrak p R_{\mathfrak p})^n$ for some $n\in \Z$.
    Define $\nu_{\mathfrak p}(\mathfrak a) = n$.
\end{definition}
\begin{remark}
    It follows from proposition \ref{prop:finite_prime} that the numbers $\nu_{\mathfrak p}(\mathfrak a)$ are almost all $0$.
\end{remark}
\begin{thm}
    Let $0\ne \mathfrak a$ be a fractional ideal.
    Then
    \begin{equation}
        \mathfrak a = \prod_{\mathfrak p}\mathfrak p^{\nu_{\mathfrak p}(\mathfrak a)}.
    \end{equation}
\end{thm}
\begin{proof}
    Localise.
\end{proof}
\begin{lemma}
    Let $n_1,\dots,n_k$ be integers, $x_1,\dots,x_k\in R$, and $\mathfrak p_1,\dots,\mathfrak p_k$ be distinct prime ideals of $R$.
    Then there exists an $x\in R$ such the $\nu_{\mathfrak p_i}(x-x_i)\ge n_i$ for all $i$.
\end{lemma}
\begin{proof}
    WLOG the $n_i\ge0$.
    If $k=1$ the result is trivial.
    Suppose $k\ge 2$.
    Let $\mathfrak a = \mathfrak p_1^{n_1}+\mathfrak p_2^{n_2}\cdots\mathfrak p_k^{n_k}$.
    Then $\nu_{\mathfrak p}(\mathfrak a) = 0$ for all $\mathfrak p$ and so $\mathfrak a = R$.
    Thus we can write $x_1 = y_{(1)}+x_{(1)}$ with $y_{(1)}\in \mathfrak p_1^{n_1}$ and $x_{(1)}\in \mathfrak p_2^{n_2}\cdots\mathfrak p_k^{n_k}$.
    It follows that $\nu_{\mathfrak p_1}(x_{(1)}-x_1) \ge n_1$ and $\nu_{\mathfrak p_i}(x_{(1)})\ge n_i$ for all $i\ne 1$.
    If we define $x_{(i)}$ similarly and let $x = \sum_i x_{(i)}$ then we obtain $\nu_{\mathfrak p_i}(x-x_i) \ge n_i$ for all $i$.
\end{proof}
\begin{corollary}
    Let $n_1,\dots,n_k$ be integers, $x_1,\dots,x_k\in K = \Frac(R)$, and $\mathfrak p_1,\dots,\mathfrak p_k$ be distinct prime ideals of $R$.
    Then there exists an $x\in K$ such the $\nu_{\mathfrak p_i}(x-x_i)\ge n_i$ for all $i$, and $\nu_{\mathfrak q}(x)\ge 0$ for $\mathfrak q\ne \mathfrak p_1,\dots,\mathfrak p_k$.
\end{corollary}
\begin{proof}
    Write $x_i = a_i/s$ for $a_i\in R$ and $0\ne s\in R$.
    Let $a\in R$ be such that $\nu_{\mathfrak p_i}(a-a_i) \ge n_i+\nu_{\mathfrak p_i}(s)$ and $\nu_{\mathfrak q}(a) \ge \nu_{\mathfrak q}(s)$ for $\mathfrak q\ne\mathfrak p_1,\dots,\mathfrak p_k$.
    Such an $a$ exists since there are only finitely many $\mathfrak q$ for which $\nu_{\mathfrak q}(s) > 0$.
    Then $\nu_{\mathfrak p_i}(a/s-a_i/s) \ge n_i$ and $\nu_{\mathfrak q}(a/s)\ge0$ for all $\mathfrak q\ne\mathfrak p_1,\dots,\mathfrak p_k$.
\end{proof}
\begin{corollary}
    Let $\mathfrak a$ be a fractional ideal of $R$. Then
    \begin{equation}
        \mathfrak a = \set{k\in K:\nu_{\mathfrak p}(k) \ge \nu_{\mathfrak p}(\mathfrak a)}.
    \end{equation}
\end{corollary}
\begin{proof}
    Let $I = \set{k\in K:\nu_{\mathfrak p}(k) \ge \nu_{\mathfrak p}(\mathfrak a)}$ and let $\mathfrak p\normal R$ be a prime ideal.
    Then certainly $I_{\mathfrak p} \subseteq \mathfrak a_{\mathfrak p}$.
    Now suppose $k\in \mf a_{\mf p}$.
    To see that $k\in I_{\mf p}$ we must show that there exists an $r\in R\backslash \mf p$ such that $rk\in I$.

    Let $r\in R$ be such that $\nu_{\mf q}(r) \ge \nu_{\mf q}(\mf a) - \nu_{\mf q}(k)$ for all $\mf q\ne \mf p$ and $\nu_{\mf p}(r-1)\ge 1$.
    Such an $r$ exists because there are only finitely many $\mf q$ such that $\nu_{\mf q}(\mf a) - \nu_{\mf q}(k) \ne 0$.
    Then $\nu_{\mf p}(r) = \nu_{\mf p}(r-1+1) = 0$ and so $r\in R\backslash \mf p$.
    Moreover, since $k\in \mf a_{\mf p}$ we also have $\nu_{\mf p}(k)\ge \nu_{\mf p}(\mf a)$.
    Therefore $\nu_{\mf q}(rk)\ge \nu_{\mf q}(\mf a)$ for all $\mf q$.
    Thus $rk\in I$ and $r\in R\backslash \mf p$.
\end{proof}
\begin{corollary}
    A Dedekind domain with only finitely many prime ideals is principal.
\end{corollary}
\begin{proof}
    For every $\mf p\normal R$ there exists an $r\in R$ such that $\nu_{\mf q}(r) = \delta_{\mf p,\mf q}$.
    It follows that $\mf p = (r)$.
    Thus all the prime ideals are principal, and hence all ideals are principal (since all ideals are a product of prime ideals).
\end{proof}
\begin{remark}
    To see why such a $r$ exists, note that there exists a $\pi\in R$ such that $\nu_{\mf p}(\pi) = 1$.
    Then let $r$ be such that $\nu_{\mf p}(r-\pi)\ge 2$ and $\nu_{\mf q}(r-1) \ge 1$.
\end{remark}
\begin{corollary}
    Let $\mf p_1,\dots,\mf p_k$ be prime ideals in $R$ and $n_1,\dots,n_k\in \Z$.
    Then there exists a $x\in K := \Frac(R)$ such that $\nu_{\mf p_i}(x) = n_i$.
\end{corollary}
\begin{proof}
    Follows from the previous remark.
\end{proof}
\begin{definition}
    Let $\mf a,\mf b$ be fractional ideals of $R$. 
    We say that $\mf a\mid \mf b$ if there is an ideal $\mf c$ of $R$ such that $\mf b = \mf a\mf c$.
\end{definition}
\begin{corollary}
    Let $\mf a, \mf b$ be fractional ideals of $R$.
    Then $\mf a\mid \mf b$ iff $\mf a\supseteq \mf b$.
\end{corollary}
\begin{proof}
    $(\Rightarrow)$ Obvious.
    $(\Leftarrow)$ $\mf a\supseteq \mf b$ implies that $\nu_{\mf p}(\mf b)\ge \nu_{\mf p}(\mf a)$ for all $\mf p$.
    Thus $\nu_{\mf p}(\mf a^{-1}\mf b)\ge 0$ for all $\mf p$ and so is an ideal of $R$.
\end{proof}
\section{Extensions}
\begin{notation}
    Let $K$ be a field and $L$ a finite extension of $K$.
    Write $n$ for $\abs{L:K}$.
    Let $A$ denote a Noetherian domain with $\Frac(A) = K$.
    We denote by $B$ the integral closure of $A$ in $L$.
    Note that if we let $S = A^\times$, then $B[S^{-1}] = B.K$ is the integral closure of $K$ in $L$ i.e. the whole of $L$.
    We say that $(F)$ is satisfied if $B$ is a finitely generated $A$-module.
\end{notation}
\begin{proposition}
    Suppose hypothesis $(F)$ is satisfied.
    If $A$ is Dedekind then $B$ is Dedekind.
\end{proposition}
\begin{proof}
    It is clear that $B$ is Noetherian and normal.
    By the going up theorem $\dim(B) = \dim(A) = 1$ and so $B$ is Dedekind.
\end{proof}
\begin{definition}
    Suppose hypothesis $(F)$ is satisfied.
    Let $\mf B$ be a non-zero prime ideal of $B$.
    If $\mf p = \mf B \cap A$ then we say that $\mf B$ divides $\mf p$.
    This is a good definition because $\mf B$ divides $\mf p$ iff $\mf B$ divides $\mf pB$.
    We define $e_{\mf B} = \nu_{\mf B}(\mf pB)$ and call it the ramification index of $\mf B$ in the extension $L/K$.
    Note that 
    \begin{equation}
        \mf pB = \prod_{\mf B\mid \mf p}\mf B^{e_{\mf B}}.
    \end{equation}

    If $\mf B$ divides $\mf p$ then note that $B/\mf B$ is a field extension of $A/\mf p$.
    We call the degree of the extension $f_{\mf B}$.
    This number is finite as $B$ is finitely generated over $A$.
    When there is only one prime ideal $\mf B$ dividing $\mf p$ and $f_{\mf b} = 1$ we say that $L/K$ is \textit{totally ramified} at $\mf p$.
\end{definition}
\begin{thm}
    Suppose hypothesis $(F)$ is satisfied.
    Let $\mf p$  be a non-zero prime ideal of $A$. 
    The ring $B/\mf pB$ is a $A/\mf p$-algebra of degree $n$ isomorphic to $\prod_{\mf B\mid \mf p}B/\mf B^{e_{\mf B}}$.
    Moreover 
    \begin{equation}
        n = \sum_{\mf B\mid \mf p}e_{\mf B}f_{\mf B}.
    \end{equation}
\end{thm}
\begin{proof}
    The isomorphism follows from the Chinese remainder theorem.
    Note that $A\cap \mf pB = \mf p$ and so $A/\mf p\hookrightarrow B/\mf pB$.
    It follows from \ref{prop:q_dim} that the dimension of $B/\mf pB$ as an $A/\mf p$ vector space is $\sum_{\mf B\mid \mf p}e_{\mf B}f_{\mf B}$.
    It remains to show that this number is $n$.
    Let $S = A\backslash \mf p$ and write $A' = A[S^{-1}]$ and $B' = B[S^{-1}]$.
    We have that $A/\mf pA \cong A'/\mf pA'$.
    To see that $B'/\mf pB'$ note that $S$ maps to $(A/\mf p)^\times$ in $B/\mf pB$.
    By proposition \ref{prop:DVR_closure}, $B'$ is a free $A'$ module of rank $n$.
    It follows that $B/\mf pB \cong B\otimes_AA/\mf p$ is a dimension $n$ $A/\mf p$ vector space.
\end{proof}
\begin{remark}
    Let $\mf B\mid \mf p$ and $x\in K$.
    Then $\nu_{\mf B}(x) = e_{\mf B}\nu_{\mf p}(x)$.
    This follows from the fact that $\nu_{\mf B}(x) = \nu_{\mf B}(xB)$ and $\nu_{\mf p}(x) = \nu_{\mf p}(xA)$.
    We say that the valuation $\nu_{\mf B}$ prolongs or extends $\nu_{\mf p}$ with index $e_{\mf B}$.
\end{remark}
\begin{proposition}
    Let $w$ be a discrete valuation on $L$ extending $\nu_{\mf p}$ with index $e$.
    Then there is a prime divisor $\mf B$ of $\mf p$ with $w = \nu_{\mf B}$ and $e = e_{\mf B}$.
\end{proposition}
\begin{proof}
    Let $W$ be the ring of $w$ and $\mf q$ its maximal ideal.
    Since $w$ extends $\nu_{\mf p}$, $A\subseteq W$.
    Since $W$ is integrally closed in $L$, $B\subseteq W$.
    Let $\mf B = \mf q \cap B$.
    Then it is clear that $\mf B \cap A = \mf p$ and so $\mf B$ divides $\mf p$.
    Moreover $B_{\mf B}\subseteq W$.
    But $B_{\mf B}$ is a maximal subring of $L$ and so $W = B_{\mf B}$.
    Thus $w = \nu_{\mf B}$ and $e = e_{\mf B}$.
\end{proof}
\subsection{Hypthesis (F)}
\begin{proposition}
    Hypothesis $(F)$ is satisfied if $L/K$ is a separable extension.
\end{proposition}
\begin{proof}
    $\Tr:L\to K$ gives rise to a symmetric non-degenerate $K$-bilinear form on $L$ by composing it with the mulitplication map.
    Note that if $x\in B$ then $\Tr(x)\in A$ by integrality considerations.

    Now let $\set{e_i}_i$ be a basis of $L$ over $K$ with the $e_i\in B$ (this exists because $B.K = L$) and let $V$ be the free $A$-module spanned by the $e_i$'s.
    For every $A$-submodule $M$ of $L$, define 
    \begin{equation}
        M^* = \set{x\in L: \Tr(xy)\in A\ \forall y\in M}.
    \end{equation}
    Clearly $V\subseteq B \subseteq B^* \subseteq V^*$.
    Since $A$ is Noetherian it thus suffices to show that $V^*$ is a finitely generated $A$-module.
    Let $\set{f_i}$ be the dual basis of $\set{e_i}$ with respect to $\Tr$.
    Then it is clear that $V^*$ is the free $A$-module generated by the $f_i$.
    This completes the proof.
\end{proof}
\section{Norm and Inclusion}
\begin{definition}
    Let $A,B$ be as in the previous section and write $I_A,I_B$ for their respective ideal groups.
    Define the homomorphisms $i:I_A\to I_B, N:I_B\to I_A$ by $i(\mf p) = \mf p B = \prod_{\mf B\mid \mf p}\mf B^{e_{\mf B}}$ and $N(\mf B) = \mf p^{f_{\mf B}}$ where $\mf B\mid \mf p$ and extend linearly.
\end{definition}
\subsection{Grothendieck Ring}
\begin{proposition}
    Let $\mf a$ be a fractional ideal of $A$ and $\mf p\normal A$ a non-zero prime ideal of $A$.
    Then $\mf a/\mf p\mf a \cong A/\mf p$ as $A$-modules.
\end{proposition}
\begin{proof}
    Write $\mf a = \prod_{i = 1}^n\mf p_i^{a_i}$ and wlog $\mf p = \mf p_1$.
    Let $x\in K:= \Frac(A)$ be such that $\nu_{\mf p_i}(x) = a_i$.
    Then $x\in \mf a\backslash \mf p \mf a$.
    Now define the map $A\to \mf a/\mf p\mf a$ by $r\mapsto rx$.
    It clearly descends to $A/\mf p$ and so we obtain a map $\phi:A/\mf p\to \mf a/\mf p\mf a$.
    Since $\mf p$ is maximal the kernel must be trivial.
    For surjectivity note that $\mf p\mf a\subsetneq \mf p\mf a+ Ax \subseteq \mf a$ and so by looking at the prime factorisation we must have $\mf p \mf a + Ax = \mf a$.
\end{proof}
\begin{definition}
    Let $\mathcal C_A$ be the category of $A$-modules of finite length.
    This is an abelian category so let $R(\mathcal C_A)$ denote the Grothendieck group of $\mathcal C_A$.

    If $M\in \mathcal C_A$ we have a composition series $0=M_0\subseteq M_1\subseteq \cdots\subseteq M_m = M$.
    The quotients $M_i/M_{i-1}$ are isomorphic to $A/\mf p_i$ for some $\mf p_i$ a non-zero prime ideal of $A$.
    Define $\chi_A(M) = \prod_i\mf p_i \in I_A$.
    This is well defined by the Jordan-Holder theorem.
    Moreover it is clear that $\chi_A:(R(\mathcal C_A),+) \to I_A$ is a homomorphism.
\end{definition}
\begin{proposition}
    Let $\mf a\subseteq \mf b$ be fractional ideals of $A$.
    Then $\chi_A(\mf b/\mf a) = \mf a\cdot \mf b^{-1}$.
    In particular $\chi_A(A/\mf a) = \mf a$ for ideals $\mf a\normal A$.
\end{proposition}
\begin{proof}
    Take the obvious composition series.
\end{proof}
\begin{corollary}
    $\chi_A$ is an isomorphism.
\end{corollary}
\begin{remark}
    Note that $\mathcal C_B$ naturally includes into $\mathcal C_A$.
\end{remark}
\begin{proposition}
    Let $M\in \mathcal C_B$. 
    Then $\chi_A(M) = N(\chi_B(M))$.
    In other words
    \begin{equation}
        \begin{tikzcd}
            I_B \arrow[r,"N"] & I_A \\
            R(\mathcal C_B) \arrow[r] \arrow[u,"\chi_B"] & R(\mathcal C_A) \arrow[u,"\chi_A"]
        \end{tikzcd}
    \end{equation}
    commutes.
\end{proposition}
\begin{remark}
    Tensoring by $B$ gives a functor $\otimes_AB: \mathcal C_A\to \mathcal C_B$.
\end{remark}
\begin{proposition}
    $\otimes_AB$ is an exact functor.
\end{proposition}
\begin{proof}
    Suffices to check locally. 
    But $B_{\mf p}$ is a free $A_{\mf p}$-module.
\end{proof}
\begin{proposition}
    The following diagram commutes
    \begin{equation}
        \begin{tikzcd}
            I_A \arrow[r,"i"] & I_B \\
            R(\mathcal C_A) \arrow[r,"R(\otimes_AB)"] \arrow[u,"\chi_A"] & R(\mathcal C_B). \arrow[u,"\chi_B"]
        \end{tikzcd}
    \end{equation}
\end{proposition}
\begin{proof}
    Suffices to check for $A/\mf p$.
    But $i(\chi_A(A/\mf p)) = \mf pB = \chi_B(B/\mf pB)$.
\end{proof}
\begin{proposition}
    Let $x\in L$.
    Then $N(xB) = N_{L/K}(x)A$.
\end{proposition}
\section{Simple extensions}
Let $A$ be a local ring with maximal ideal $\mf m$ and residue field $k$, $f\in A[x]$ and $B_f := A[x]/f$.
We have
\begin{equation}
    \bar B_f := B_f/\mf mB_f \cong A[x]/(\mf m, f) \cong k[x]/(\bar f)
\end{equation}
where $\bar f$ is the reduction of $f$ to $k[x]$.
Let $\bar f = \prod_i\phi_i^{e_i}$ be the decomposition of $\bar f$ into irreducibles, and for each $\phi$ pick a $g_i\in A[x]$ such that $\bar g_i = \phi_i$.
Write $\mf m_i$ for the ideal in $B_f$ generated by $\mf m$ and the image of $g_i$ in $B_f$.
\begin{proposition}
    The ideals $\mf m_i$ are maximal and distint and are all the maximal ideals of $B_f$.
    Moreover $B_f/\mf m_i \cong k_i := k[x]/(\phi_i)$.
\end{proposition}
\begin{proof}
    $\mf m_i$ is the preimage of $(\phi_i)/(f)\normal \bar B_f$ in $B_f$.
    Thus $B_f/\mf m_i \cong k_i$ and so the $\mf m_i$ are distinct and maximal.
    To show that every maximal ideal is one of the $\mf m_i$ it suffices to show that if $\mf n$ is maximal it contains $\mf m$.
    If not then $\mf m B_f+\mf n = B_f$ which would imply that $\mf n = B_f$ (by Nakayama) which is a contradiction.
\end{proof}
\subsection{Unramified case}
Suppose $\bar f$ is irreducible and $A$ is a DVR.
Then $B_f$ is a DVR with maximal ideal $\mf m B_f$ and residue field $k[x]/(\bar f)$.
\section{Galois Extensions}
Let $A,B$ be as before, but suppose additionally that $L/K$ is a Galois extension.
Write $G(L/K)$ for the Galois gorup of the extension.
\begin{proposition}
    $G(L/K)$ acts transitively on the set of primes in $B$ dividing $\mf p\normal A$.
\end{proposition}
\begin{corollary}
    Let $\mf p$ be a non-zero prime ideal of $A$.
    Then integers $e_{\mf B}$ and $f_{\mf B}$ (for $\mf B\mid \mf p$) only depend on $\mf p$.
    If we denote these numbers $e_{\mf p}, f_{\mf p}$, and $g_{\mf p}$ for the number of prime ideals in $B$ dividing $\mf p$ then $n = e_{\mf p}f_{\mf p}g_{\mf p}$.
\end{corollary}
\begin{definition}
    Define $D_{\mf B}(L/K) = \stab_{G(L/K)}(\mf B)$.
\end{definition}
\chapter{Appendix - Commutative algebra results}
\section{Prime ideals}
\begin{lemma}
    \label{lemma:prime_avoidance}
    (Prime avoidance lemma).
    Let $I\normal R$ be an ideal and let $P_1,\dots,P_k\normal R$ be prime ideals.
    If $I \subseteq \cup_iP_i$ then there exists an $i$ such that $I\subseteq P_i$.
\end{lemma}
\section{Localisation}
\begin{proposition}
    Let $\mathfrak a,\mathfrak b \normal R$ be ideals and $\mathfrak p \normal R$ a prime ideal.
    Then 
    \begin{equation}
        (1)\ (\mathfrak a + \mathfrak b)_{\mathfrak p} = \mathfrak a_{\mathfrak p} + \mathfrak b_{\mathfrak p}, \quad (2)\ (\mathfrak a . \mathfrak b)_{\mathfrak p} = \mathfrak a_{\mathfrak p} . \mathfrak b_{\mathfrak p}, \quad (3)\ (\mathfrak a : \mathfrak b)_{\mathfrak p} = (\mathfrak a_{\mathfrak p} : \mathfrak b_{\mathfrak p})
    \end{equation}
    where $(3)$ additionally requires $\mathfrak b$ to be finitely generated.
\end{proposition}
\begin{proof}
    $(1)$ and $(2)$ are trivial.
    For $(3)$ note that $(\subseteq)$ always holds.
    Now suppose $\mathfrak b = (b_1,\dots,b_k)$.
    If $r/s\in(\mathfrak a_{\mathfrak p}:\mathfrak b_{\mathfrak p})$ then so is $r/1$.
    Thus we have $(r/1)(b_i/1)=a_i/s_i$ for some $a_i\in\mathfrak a,s_i\not\in\mathfrak p$.
    Thus there is a $u_i\not\in\mathfrak p$ such that $(u_is_i)(rb_i)=u_ia_i\in\mathfrak a$.
    Let $u = \prod_i(u_is_i)$.
    Then $(ur)b_i\in\mathfrak a$ for all $i$.
    Thus $ur\in (\mathfrak a:\mathfrak b)$ and so $r/1 = ur/u \in (\mathfrak a:\mathfrak b)_{\mathfrak p}$ and so $r/s$ is too.
\end{proof}
\begin{proposition}
    Let $\mathfrak a,\mathfrak b$ be $R$-submodules of $K = \Frac(R)$ and $\mathfrak p \normal R$ a prime ideal.
    Then 
    \begin{equation}
        (1)\ (\mathfrak a + \mathfrak b)_{\mathfrak p} = \mathfrak a_{\mathfrak p} + \mathfrak b_{\mathfrak p}, \quad (2)\ (\mathfrak a . \mathfrak b)_{\mathfrak p} = \mathfrak a_{\mathfrak p} . \mathfrak b_{\mathfrak p}, \quad (3)\ (\mathfrak a : \mathfrak b)_{\mathfrak p} = (\mathfrak a_{\mathfrak p} : \mathfrak b_{\mathfrak p})
    \end{equation}
    where $(3)$ additionally requires $\mathfrak b$ to be finitely generated and $(\mathfrak a:\mathfrak b) := \set{k\in K:k\mathfrak b \subseteq \mathfrak a}$.
\end{proposition}
\begin{proof}
    First note that $\mathfrak a_{\mathfrak p} = \set{\frac1sk:s\in R\backslash\mathfrak p, k\in \mathfrak a}$.
    Then $(1)$ and $(2)$ follow.
    For $(3)$, $(\subseteq)$ always holds.
    If $\mathfrak b$ is additionally finitely generated, let $k_1,\dots,k_n$ be its generators.
    For $k\in (\mathfrak a_{\mathfrak p}:\mathfrak b_{\mathfrak p})$, we have that for each $i$ there are $a_i\in\mathfrak a, s_i\not\in\mathfrak p$ such that $kk_i = \frac1{s_i}a_i$.
    Let $s = \prod_is_i$. Then $skk_i\in\mathfrak a$ for all $i$.
    Thus $sk\in(\mathfrak a:\mathfrak b)$ and so $k\in(\mathfrak a:\mathfrak b)_{\mathfrak p}$.
\end{proof}
\section{Integrality}
\begin{thm}
    \label{thm:int_dep}
    Let $S\subseteq R$ be rings and $x\in R$.
    The following are equivalent:
    \begin{enumerate}
        \item $x$ is integral over $S$
        \item $S[x]$ is a finite $S$-algebra
        \item $S[x]$ is contained ina subring $T\subseteq R$ that is a finite $S$-alg
        \item There is an $S[x]$-module $M$ such that $\Ann_{S[x]}(M) = 0$ and $M$ is finitely generated as a $S$-module.
    \end{enumerate}
\end{thm}
\begin{lemma}
    Let $S\subseteq R$ be an integral extension of rings. 
    Let $Q\normal R$ be a prime ideal, and set $P = Q\cap S$.
    Then $Q$ is maximal in $R$ iff $P$ is maximal in $S$.
\end{lemma}
\begin{lemma}
    (Incomparability).
    Let $S\subseteq R$ be an integral extension of rings.
    Let $Q,Q'\normal R$ be prime ideals with $Q\subseteq Q'$ and $Q\cap S = Q'\cap S$. 
    Then $Q = Q'$.
\end{lemma}
\begin{lemma}
    (Lying over).
    Let $S\subseteq R$ be an integral extension of rings.
    Let $P\normal S$ be prime.
    Then there exists a prime $Q\normal R$ such that $Q\cap S = P$.
\end{lemma}
\begin{thm}
    \label{thm:going_up}
    (Going up).
    Let $S\subseteq R$ be an integral extension of rings.
    Let $P_1\subseteq P_2\subseteq \cdots \subseteq P_n$ be a chain of primes in $S$ and $Q_1\subseteq Q_2\subseteq \cdots \subseteq Q_m$ with $0\le m < n$ be a chain of primes in $R$ with $Q_i\cap S = P_i$.
    Then the chain can be extended to $Q_1\subseteq Q_2\subseteq \cdots \subseteq Q_n$ with $Q_i\cap S = P_i$.
\end{thm}
\begin{thm}
    \label{thm:going_down}
    (Going down).
    Let $S\subseteq R$ be an integral extension of domains such that $S$ is normal.
    Let $P_1\supseteq P_2\supseteq \cdots \supseteq P_n$ be a chain of primes in $S$ and $Q_1\supseteq Q_2\supseteq \cdots \supseteq Q_m$ with $0\le m < n$ be a chain of primes in $R$ with $Q_i\cap S = P_i$.
    Then the chain can be extended to $Q_1\supseteq Q_2\supseteq \cdots \supseteq Q_n$ with $Q_i\cap S = P_i$.
\end{thm}
\begin{proposition}
    Let $A\subset R$ be rings and $S$ a multiplicativly subset of $A$.
    If $B$ is the integral closure of $A$ in $R$ then $B[S^{-1}]$ is the integral closure of $A[S^{-1}]$ in $R[S^{-1}]$.
\end{proposition}
\section{Noetherian Local Rings}
\begin{thm}
    \label{thm:noeth_loc_ring}
    Let $R$ be a Noetherian local ring with maximal ideal $\mathfrak m$.
    Then $\cap_n\mathfrak m^n = 0$.
\end{thm}
\section{Miscellaneous Results on DVRs and Dedekind domains}
\begin{proposition}
    Let $R$ be a Dedekind domain and $\mf p\normal R$ be a prime ideal.
    Then for any $n\ge 1 $, 
    \begin{equation}
        R/\mf p^n \cong R_{\mf p}/\mf p_{\mf p}^n.
    \end{equation}
\end{proposition}
\begin{proof}
    We have the short exact sequence
    \begin{equation}
        0 \to \mf p^n \to R \to R/\mf p^n \to 0.
    \end{equation}
    Localising at $\mf p$ we get that 
    \begin{equation}
        R_{\mf p}/\mf p_{\mf p}^n \cong (R/\mf p^n)_{\mf p/\mf p^n}
    \end{equation}
    and so it suffices to show that all the elements not in $\mf p/\mf p^n$ are invertible in $R/\mf p^n$.
    For this let $r\in R\backslash \mf p$.
    Then $(r) + \mf p = R$ and so $(r) + \mf p^n = R$.
    It follows that $r+\mf p^n$ is invertible in $R/\mf p^n$.
\end{proof}
\begin{proposition}
    \label{prop:q_dim}
    Let $R$ be a DVR with maximal ideal $\mf m$ and let $k$ be a field in $R/\mf m^n$.
    Then $\dim_k(R/\mf m^n) = n\cdot\abs{R/\mf m:k}$.
\end{proposition}
\begin{proof}
    We have the following chain of ideals, and hence $k$ vector spaces
    \begin{equation}
        0 \subseteq \mf m^{n-1}/\mf m^n \subseteq \cdots \subseteq \mf m/\mf m^n \subseteq R/\mf m^n.
    \end{equation}
    It follows that $\dim_k(R/\mf m^n) = \sum_{i = 0}^{n-1} \dim_k(\mf m^i/\mf m^{i+1})$.
    But $\mf m^i/\mf m^{i+1} \cong \mf m^i\otimes_R R/\mf m \cong R/\mf m$ since $\mf m^i\cong R$ as $R$-modules.
    As $k$ naturally lives inside $R/\mf m$ the result follows.
\end{proof}
\begin{proposition}
    \label{prop:DVR_closure}
    Let $R$ be a DVR and $K = \Frac(R) \le L$ be a finite field extension.
    Let $S$ be the integral closure of $R$ in $L$, and suppose that $S$ is a finitely generated $R$-module.
    Then $S$ is a free module of rank $\abs{L:K}$.
\end{proposition}
\begin{proof}
    $R$ is a PID so $S$ is a free $R$-module.
    Suppose $\alpha_1,\dots,\alpha_k$ are an $R$-basis for $S$.
    Since $K.S = L$, $\alpha_1,\dots,\alpha_k$ span $L$ as a $K$ vector space.
    But since $K = \Frac(R)$, they are also linearly independent over $K$ and so $k = \abs{L:K}$.
\end{proof}
\begin{proposition}
    Let $R$ be a DVR.
    Then $R$ is a maximal subring of $\Frac(R)$.
\end{proposition}
\begin{proof}
    Obvious.
\end{proof}
\end{document}

